\documentclass[a4paper]{report}

\usepackage[utf8]{inputenc}
\usepackage[english,italian]{babel}
\usepackage[hyphens]{url}
\usepackage{hyperref}
\usepackage{graphicx}
\usepackage{lipsum}

\begin{document}
	\begin{titlepage}
		\begin{center}
			\includegraphics[width=\textwidth]{Logo.jpg}\\
			{\large{\bf Corso di Laurea Magistrale in Informatica}}
		\end{center}
	
		\vspace{14mm}
		\begin{center}
			{\LARGE{\bf MEZZI PUBBLICI, MEZZI PRIVATI E}}\\
			\vspace{3mm}
			{\LARGE{\bf CAR SHARING A CONFRONTO}}\\
			\vspace{4mm}
			{\LARGE{\bf NELL'AREA DI MILANO}}\\
		\end{center}
	
		\vspace{14mm}
		\begin{center}
			{\large{\bf Tesi di Laurea di}}\\
			\vspace{3mm}
			{\Large{\bf MAURO MASTRAPASQUA}}\\
			\vspace{2mm}
			{\large{\bf Matricola 892629}}\\
		\end{center}
	
		\vspace{14mm}
		\begin{flushleft}	{\large
				prova
			}
			{\normalsize{\bf Relatore}}\\
			\vspace{1mm}
			{\large{\bf Alessandro Magno}}\\
			\vspace{4mm}
			{\normalsize{\bf Correlatore}}\\
			\vspace{1mm}
			{\large{\bf Carlo Magno}}\\
		\end{flushleft}
	
		\vspace{14mm}
		\begin{center}
			{\large{\bf Anno Accademico 4000 a.C}}
		\end{center}
	\end{titlepage}

	\tableofcontents
	
	\chapter{Introduzione}
	{\large
		prova
	}
	TCP/IP over Avian Carriers\cite{waitzman1990standard}

	\chapter{Confronto}
	{\large
		L'obiettivo di questo studio è di mettere a confronto in termini di velocità, prezzo, rapporto velocità/prezzo e altri parametri i diversi mezzi di trasporto a disposizione nel Comune di Milano per spostarsi al suo interno. Per quanto sia un dato di fatto che un'automobile vada più veloce di un tram o che quest'ultimo vada più veloce di una persona a piedi, questi fatti potrebbero non sussistere all'interno di una città come Milano per via delle proprietà intrinsiche che accomunano ogni città di medie e grandi dimensioni: traffico, deviazioni di percorso, incidenti, eventi di massa e ore di punta.
	}

	{\large
		L'idea alla base del confronto è quella di simulare in tempo reale, ripetutamente a intervalli regolari, la percorrenza di alcune tratte prestabilite lungo l'arco della giornata, per un arco di tempo di qualche mese, dove per ogni simulazione di tratta vengono calcolate diverse soluzioni ognuna con un mezzo di trasporto diverso. I dati collezionati da questa simulazione mi permetterebbero di fare le dovute analisi sui termini di paragone scelti per confrontare i mezzi, traendo infine le conclusioni.
	}

		\section{Mezzi a confronto}
		{\large
			I mezzi di trasporto presi in considerazione per questo confronto sono: servizio di trasporto pubblico ATM (include tram, metro, bus, filobus e passanti Trenord), bicicletta di proprietà, macchina di proprietà e servizi di car sharing ShareNOW (precedentemente chiamata CAR2GO) (Smart for2, Smart for4), Enjoy (Fiat 500) e Sharengo (veicoli elettrici).
		}
	
		\section{Stima in tempo reale}
		{\large
			TODO
		}
	
	
	\chapter{Cap3}
	\chapter{Cap4}

	\bibliographystyle{plain}
	\bibliography{Biblio}
\end{document}
