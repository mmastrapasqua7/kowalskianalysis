Quando ci si deve spostare all'interno di una città, spesso ci si affida a servizi di navigazione per ottenere informazioni sul percorso più veloce da intraprendere.
Il mezzo di trasporto nella maggior parte dei casi viene scelto a priori dall'utente per diversi motivi quali abitudine, differenza di prezzo o mancanza di interesse per le alternative, anche se tale mezzo dovesse risultare il più lento.

Sebbene sia ragionevole pensare che l'auto di proprietà sia più veloce dei mezzi pubblici e delle biciclette, questo fatto potrebbe non sussistere all'interno di una città come Milano. Infatti, secondo l'ultimo studio annuale di TomTom del 2019\cite{tomtomindexmilan}, Milano rientra tra le prime 100 città su scala globale per livello di congestione stradale, con un traffico tale da far quasi raddoppiare i tempi di percorrenza in auto durante le ore di punta del mattino e del tardo pomeriggio. In aggiunta al fatto che alcuni mezzi non soffrono minimamente di questo problema, come le metropolitane, o ne soffrono parzialmente, come le biciclette, questo rende l'ipotesi di partenza più discutibile.

.\newline
...studiare altri papers... \newline
.

Dato che lo stato attuale della tecnologia permette ai servizi di navigazione di avere una stima del tempo di percorrenza molto precisa e realistica, grazie a dati in tempo reale su traffico, incidenti, deviazioni di percorso e grazie allo storico delle tratte percorse dalle loro migliaia di utenti, si è deciso di basare questo studio su tali valori delle stime, con l'intento di capire dal punto di vista di un qualsiasi utente, in un qualsiasi punto della mappa e a una qualsiasi ora del giorno, se esiste un mezzo ideale per affrontare ogni tipo di viaggio all'interno della città di Milano e di analizzarne eventuali variazioni.