\section{Riflessioni sul risultato}

Il risultato ottenuto dal confronto tra bicicletta e servizio di car sharing Enjoy sulla base dell'orario risulta molto interessante sotto diversi aspetti. Il primo è sicuramente la relazione inversamente proporzionale del grafico stesso, \ref{image:14}, con quello delle prestazioni del car sharing nel grafico \ref{image:7}, che vede la frequenza relativa delle vittorie della bicicletta aumentare al diminuire della velocità media in car sharing dovute al traffico stradale. Questa relazione cattura la grande influenza della congestione stradale sull'efficienza in termini di tempo dell'uso dell'auto, sia essa di proprietà o condivisa. L'influenza del traffico sul servizio di car sharing preso in considerazione ha fatto sì che nelle ore intorno alle 8:00 e ancora intorno alle 18:00, lo stesso percorso è stato coperto in bicicletta nello stesso tempo se non più velocemente il 45\% delle volte, quasi 1 volta su 2. Il car sharing dunque sembra risolvere solamente il primo e il secondo problema delle automobili citato nell'introduzione riguardo l'inquinamento e il costo, ma il problema dell'efficienza è maggiormente amplificato dal tempo impiegato a raggiungere un'auto libera a piedi, stimato di circa 6 minuti e mezzo di media, a tal punto che una bicicletta è in grado di pareggiare le sue performance negli orari di picco del traffico.

All'atto pratico però non risulta fattibile sostituire il servizio di car sharing o dei mezzi pubblici per fare così tanti chilometri pedalando, dato che si tratta di un'attività fisica medio-intensa e che può risultare stancante se fatta per più di 20 minuti. Tuttavia, per effettuare questo studio non si è preso in considerazione uno dei mezzi alternativi in gran diffusione nelle più grandi città del mondo, diretto concorrente della bicicletta e delle sue varianti a pedalata assistita: il monopattino elettrico.
Il monopattino elettrico infatti ha una velocità massima di 25 km/h, contro i 15 km/h di una pedalata normale in bicicletta, e può essere comprato per un costo che va dai 500 euro in sù oppure noleggiato (in mobilità condivisa) da uno dei tanti provider come Helbiz per €0.25 al minuto previo €1.00 di sblocco. Risulta molto più pratico da distribuire nella città e difatti sono molto diffusi, con un tempo medio per trovarne uno inferiore a quello del car sharing. Visti i suoi pregi e la sua velocità più alta rispetto alla bici, non sarebbe improbabile che tale mezzo riesca a battere il car sharing negli orari di punta più del 50\% delle volte e aumentando la percentuale anche nelle ore con meno traffico. Va inoltre fatta una nota riguardo l'economicità in caso di utilizzo di tale mezzo in mobilità condivisa: nel car sharing, si è presa in considerazione la lista delle auto libere solo ed esclusivamente di un servizio, Enjoy, di fatto aumentando il tempo di ricerca di un auto in mobilità condivisa in generale, per via di una quota fissa da pagare annualmente a ciascuno dei provider che si intende utilizzare, il che sposta la scelta del servizio al momento dell'iscrizione che per la maggior parte dei casi ricade su un unico provider. Al contrario, per i servizi sharing del monopattino non è richiesta nessuna documentazione ne quota fissa da pagare annualmente, ma si paga solo l'utilizzo.

\section{Possibili estensioni}

Risulterebbe interessante rifare tale studio usando dei servizi che forniscono dati in tempo reale per tutti i mezzi di trasporto usati in questo studio, aggiungendo anche quelli non considerati, come i taxi, i monopattini elettrici in sharing e non, le biciclette in sharing e soprattutto le moto. Si potrebbe, inoltre, mettere a confronto i costi singoli di ogni viaggio per determinare la spesa complessiva da parte dell'utente delle diverse opzioni a disposizione.

Avendo a disposizione un dataset di viaggi realmente compiuti, o stimati come nel caso di questo studio, risulterebbe molto utile socialmente la scrittura di un software in grado di sfruttare tale conoscenza per consigliare il miglior mezzo di trasporto per affrontare tratte quotidiane o regolarmente percorse. Per esempio, sulla base di un input contenente una tratta regolarmente percorsa, indicando il punto di partenza, il punto di arrivo insieme al giorno e l'orario in cui viene affrontata abitualmente, un possibile software potrebbe effettuare le stesse statistiche di questo studio per fornire infografiche contenenti dati quali il tempo medio per raggiungere i mezzi in car sharing dal punto di partenza indicato, il mezzo più veloce, il più economico o il migliore secondo una media pesata, lasciando all'utente la libertà di scegliere le priorità. Un software del genere infatti potrebbe evidenziare che il mezzo usato abitualmente dall'utente in determinati giorni a dei determinati orario risultano molto inefficienti, spingendolo a valutare altre opzioni per spostarsi. Ovviamente un sistema del genere richiederebbe un raffinamento dei calcoli usati per stimare diverse proprietà oltre che a un database ottimizzato per l'indicizzazione dei dati storici. Progetti simili sono già stati modellati in diversi studi dove la stima del tempo di percorrenza è fatta effettuata su dati storici oppure interpolando dati storici con quelli in tempo reale forniti da utenti, e comparata con viaggi realmente effettuati, portando a ottimi risultati riguardo il livello di accuratezza della predizione ed evidenziando l'efficacia dei dati storici per le stime durante le ore di picco del traffico\cite{deeshma2014}\cite{chien2003}.
