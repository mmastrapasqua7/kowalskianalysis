L'idea alla base di questo studio è stata, dunque, quella di confrontare percorsi comuni, ovvero con punto di partenza, di destinazione e di orario di partenza, sulla base del loro tempo di percorrenza ed evincere 


















Lo stato attuale della tecnologia permette ai servizi di navigazione di avere una stima del tempo di percorrenza molto precisa e realistica, grazie a dati in tempo reale su traffico, incidenti, deviazioni di percorso e grazie allo storico dei tragitti percorsi dalle loro migliaia di utenti. Sfortunatamente, tali servizi non rendono pubblico tale storico, ma si limitano a pubblicare qualche analisi fatta su di esso. Inoltre, molti dei servizi sono specializzati nel fornire soluzioni per un solo mezzo.

\cite{croci2014}

\cite{rotaris2010}

\cite{rotaris2019}

\cite{meinardi2008}

\todo{atrent: da qualche parte ci sta una parentesi sui sistemi di navigazione crowd (es. waze) con cronistoria delle acquisizioni}

L'idea alla base di questo studio è stata quella di creare uno storico delle tratte tramite un'utenza fittizia, simulata scegliendo a random dei punti di partenza e destinazione, inoltrando tali punti come oggetto di richiesta per ogni servizio di navigazione corrispondente a un mezzo di trasporto diverso, al fine di confrontare a posteriori le performance di ogni mezzo e analizzarne le eventuali variazioni lungo l'arco della giornata.\todo{atrent: spiegare meglio, dichiarare cosa vuoi fare, quali parametri vorrai misurare e che tipo di paragoni vorrai fare}
