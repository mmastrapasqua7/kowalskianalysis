\section{Raccolta dati}

\subsection{Le problematiche}

I dati riguardanti la simulazione sono stati raccolti a partire dall'1 marzo 2020 fino al 27 giugno 2020 per un arco di tempo di circa 4 mesi, al ritmo di 1200 richieste al giorno, che altro non sono che finte gare di percorrenza con stime dei tempi richieste ai provider.

Sebbene questo studio fosse stato pensato per un confronto in situazioni abituali di traffico e congestione caratteristiche di Milano, a influenzare questo studio è stato da subito l'emergenza epidemiologica da COVID-19. Il 22 gennaio 2020 il governo italiano ha dichiarato lo stato di emergenza, mettendo in atto le prime misure di contenimento in alcuni Comuni dove si sono verificati i primi casi del contagio. A partire dal 23 febbraio infatti il governo italiano ha emanato DPCM sempre più stringenti riguardo la circolazione e lo svolgimento delle attività commerciali, fino alla dicisione di imporre un lockdown nazionale avvenuta l'11 marzo 2020\cite{misuredelgovernopercovid}. Inoltre, nei primi giorni di utilizzo del programma scritto per eseguire le richieste tramite API, il servizio precedentemente utilizzato di Moovit ha risposto alle richieste con valori nulli o totalmente fuori scala. Per questo motivo è stato sostituito il servizio con quello offerto da Here e sono stati scartati i risultati precedentemente raccolti.

Nonostante lo studio sia stato fortemente influenzato da questo fattore, per via della circolazione più scorrevole dati i ridotti spostamenti e per la riduzione di numero di mezzi e di orario di servizio dei mezzi pubblici ATM, si è deciso comunque di procedere, tenendo conto dell'avvenuto sul giudizio finale dei risultati.

\subsection{Filtro dati errati}

\begin{table}[H]
\centering
\begin{tabular}{ | l | r | r | }
\hline

& \textbf{1 Marzo - 12 Marzo} & \textbf{4 Maggio - 27 Giugno} \\
\hline

\textbf{Coerenti}& 1988 & 49560 \\  
\textbf{Errati} & 7962 & 18290 \\
\hline
\textbf{Totale} & 9680 & 67850 \\
\textbf{\% Errati} & 79.5 & 27.0 \\
\hline
\end{tabular}
\caption{Dati errati che contengono uno o più valori nulli nelle stime}
\label{table:1}
\end{table}

I dati raccolti dalla prima esecuzione del programma, l'1 marzo 2020, fino all'entrata in vigore dello stato di lockdown nazionale, il 12 marzo 2020, sono risultati pieni di errori, ovvero stime di percorrenza nulle. La principale causa è da attribuire al servizio di Moovit, che ha subìto un aggiornamento dell'applicazione web da cui lo scraper era stato programmato per fare richieste. Successivamente a questo evento si è cercato e trovato un servizio alternativo che sostituisse la stima del percorso coi mezzi pubblici, ovvero Here, che sfortunatamente non offre stime in tempo reale ma solo statiche. I dati raccolti nel periodo di lockdown dal 13 marzo 2020 al 3 maggio 2020 compresi sono stati scartati per via della circolazione dei mezzi pubblici e privati quasi totalmente assente, e quindi poco utile nell'obbiettivo finale dello studio di confrontare i mezzi di trasporto in situazioni di traffico caratteristico da città. Per questo motivo si è scelto di usare solamente i dati a partire dall'allentamento delle misure di restrizioni da emergenza sanitaria, ovvero quelli a partire dal 4 maggio 2020 compreso.

Anche 

I provider per le stime di percorrenza dei vari mezzi hanno talvolta restituto il valore zero. Questo fenomeno è probabilmente dovuto alla generazione a random delle coordinate di partenza e arrivo, che ha permesso richieste in qualsiasi punto della mappa, compresi punti non situati direttamente sulla strada come in giardini pubblici, condimini e altre zone private, o semplicemente per disservizi. Per un confronto alla pari sono state considerate solo le righe prive di zeri nel file CSV. Come indicato dalla tabella \ref{table:1} su circa 68000 richieste, 50000 sono prive di zeri, circa il 73\% del totale, e quindi adatte per un confronto 1 a 1 tra mezzi di trasporto. 

\begin{table}[H]
	\centering
	\begin{tabular}{ | c | c | c | c | c | c | c | c | c | }
		\hline
		Data & Orario & Partenza & Arrivo & Auto & ATM & Enjoy & Bici & Piedi \\
		\hline
	\end{tabular}
\end{table}

In partenza e arrivo sono salvate le coordinate espresse in gradi del tragitto generato, nei restanti campi le stime di percorrenza di tale tragitto espresse in minuti, insieme alla data e all'orario in cui è stata effettuata la richiesta di tale tratta.

Oltre ai dati principali sono stati salvati informazioni secondarie come il numero di auto libere Enjoy al momento della richiesta, la lunghezza aerea della tratta e la lunghezza calcolata a piedi della stessa, entrambe espresse in km.

\subsection{Distribuzione lunghezza tratte generate}

\begin{figure}[H]
\includegraphics[scale=0.8]{distribuzione_tratte}
\caption{Statistiche lunghezza tratta [km]}
\label{image:2}
\end{figure}

\begin{table}[H]
\centering
\begin{tabular}{ | l r | }
\hline
\textbf{Abs. freq.} & 49560 \\
\textbf{media} & 3.80 \\
\textbf{mediana} & 3.57 \\
\textbf{std} & 1.27 \\
\textbf{min} & 2.00 \\
\textbf{max} & 9.52 \\
\hline
\end{tabular}
\caption{Statistiche lunghezza tratta [km]}
\label{table:2}
\end{table}

Nella  figura \ref{image:2} e tabella \ref{table:2} sono riportate le statistiche riguardo la lunghezza in via aerea delle tratte generate. Si può vedere come più della metà sia lunga meno di 5 km in linea aerea, risultato ottenuto in parte dai constraint imposti nella generazione.

\section{Performance dei singoli mezzi}

\begin{table}[H]
\centering
\begin{tabular}{ | l r r r r r | }
\hline
& \textbf{Auto} & \textbf{Enjoy} & \textbf{ATM} & \textbf{Bici} & \textbf{Piedi} \\
\textbf{media}   & 19.6 & 13.5 &  8.3 & 11.9 & 4.5 \\
\textbf{mediana} & 19.5 & 13.2 &  7.9 & 12.0 & 4.5 \\
\textbf{std}     &  3.9 &  3.8 &  2.3 &  1.3 & 0.0 \\
\textbf{min}     &  6.7 &  3.0 &  3.4 &  6.5 & 4.3 \\
\textbf{max}     & 48.5 & 36.1 & 27.8 & 16.9 & 4.6 \\
\hline
\end{tabular}
\caption{Statistiche velocità media [km/h]}
\label{table:3}
\end{table}

\subsection{Auto}

\subsubsection{Velocità media di ora in ora}

\begin{figure}[H]
\includegraphics[scale=1.0]{vmedia_oraria_auto}
\caption{Velocità media in auto [km/h], di ora in ora}
\label{image:3}
\end{figure}

Una delle prime analisi effettuate per ogni mezzo è stata quella di calcolare la velocità media per ogni tragitto effettuato, con l'obbiettivo di osservare eventuali variazioni di ora in ora. Come riportato visualmente nella figura \ref{image:3}, i tragitti in macchina hanno subito una notevole variazione di velocità media nell'arco 8:00-11:00 e in quello delle 17:00-19:00. Tali risultati sono in linea con quelli dello studio di TomTomIndex\cite{tomtomindexmilan} effettuato sulla città di Milano nel 2019, che evidenzia le ore 9:00 e 18:00, anche dette rush hours, come picchi di congestione stradale dal lunedì al venerdì, con un livello di congestione rispettivamente del 70\% e del 60\%. Risultano in linea con lo studio anche le ore precedenti e successive agli orari individuati come picchi.

\subsubsection{Velocità media lunedì-venerdì e sabato-domenica}

\begin{figure}[H]
\includegraphics[scale=1.0]{vmedia_oraria_auto_weekend}
\caption{Velocità media in auto [km/h], di ora in ora}
\label{image:4}
\end{figure}

La figura \ref{image:4} mostra il risultato di una ripartizione dei dati effettuata sulla base del giorno della settimana, in particolare sono stati divisi i tragitti effettuati dal lunedì al venerdì da quelli del sabato e domenica di ogni settimana. Si può notare una notevole differenza di velocità media indistintamente dall'orario di circa 2 km/h. La variazione nel sabato-domenica risulta meno evidente di quella del lunedì-venerdì, con le variazioni, ovvero i picchi di rallentamento, che si spostanto nelle fasce orarie 10:00-12:00 e 16:00-19:00. Anche questi dati risultano in linea con quelli dello studio di TomTomIndex\cite{tomtomindexmilan}, che vede un minor livello di congestione nel weekend con dei picchi nelle ore 10:00 e 18:00.

\subsubsection{Velocità media settimana dopo settimana da fine lockdown}

\begin{figure}[H]
\includegraphics[scale=1.0]{vmedia_oraria_auto_weeks}
\caption{Velocità media in auto [km/h], di ora in ora}
\label{image:5}
\end{figure}

La figura \ref{image:5} mostra il risultato di una ripartizione dei dati effettuata in base alla settimana, in particolare sono stati partizionati a gruppi di 2 settimane consecutive i tragitti effettuati a partire dal 4 maggio 2020, primo giorno dell'allentamento delle restrizioni imposte dal governo italiano per l'emergenza covid\cite{dpcm26aprile}. Nel grafico risulta evidente una degradazione della velocità media generale e lineare rispetto al passare delle settimane. Si nota inoltre che le curve corrispondenti alle settimane successive al 17 maggio, ovvero dopo le prime 2 settimane, presentino dei flessi sempre più accentuati in prossimità delle rush hours evidenziate dal grafico \ref{image:3}.

\subsubsection{Velocità media in base alla lunghezza tratta}

\begin{figure}[H]
\includegraphics[scale=0.75]{heatmap_auto}
\caption{Velocità media in auto [km/h] in base alla lunghezza tratta, di ora in ora}
\label{image:6}
\end{figure}

\todo{MM: commenti...}

\subsection{Enjoy}

\subsubsection{Velocità media di ora in ora}

\begin{figure}[H]
\includegraphics[scale=1.0]{vmedia_oraria_enjoy}
\caption{Velocità media usando Enjoy [km/h], di ora in ora}
\label{image:7}
\end{figure}

Siccome le stime di percorrenza col servizio Enjoy sono basate sul servizio di Waze, in particolare il tragitto dalla posizione dell'auto alla destinazione del tragitto, non si riscontrano particolari differenze riguardo i flessi nell'andamento della velocità media di ora in ora osservati nella figura \ref{image:3} per l'auto. La curva però risulta traslata verticalmente verso il basso per via del tragitto a piedi necessario per raggiungere un'auto libera che va a influire nel calcolo. Questa traslazione, secondo la figura \ref{image:7}, risulta di circa 5 km/h indipendentemente dall'ora del giorno. Di conseguenza sono state analizzati i dati riguardanti il tragitto a piedi.

\subsubsection{Auto libere e tempo medio per raggiungerle}

\begin{table}[H]
\centering
\begin{tabular}{ | l r | }
\hline
& \textbf{ragg. auto} \\
\textbf{media}   &  6.6 \\
\textbf{mediana} &  6.0 \\
\textbf{std}     &  4.6 \\
\textbf{min}     &  0.0 \\ 
\textbf{max}     & 43.0 \\
\hline
\end{tabular}
\caption{Statistiche tempo medio per raggiungere auto libera [min]}
\label{table:4}
\end{table}

\begin{figure}[H]
	\includegraphics[scale=1.0]{tmedio_raggiungimento_auto_enjoy}
	\caption{Tempo medio per raggiungere auto libera [min], di ora in ora}
	\label{image:8}
\end{figure}

Dai calcoli è risultato che il tempo medio impiegato a raggiungere a piedi un'auto del servizio di car sharing Enjoy oscilla intorno ai 6 minuti e mezzo, come riportato dalla tabella \ref{table:4}. Anche per questa analisi i dati sono stati partizionati per i giorni da lunedì a venerdì e per sabato e domenica. Il grafico \ref{image:8} mostra una lieve differenza nel tempo medio di circa 1 minuto a favore del primo gruppo, comunemente associato ai giorni lavorativi. A differenza della curva del fine settimana che risulta mediamente stabile, la curva dei giorni lavorativi subisce un calo nel tempo medio nelle ore del primo pomeriggio.

\begin{figure}[H]
\includegraphics[scale=1.0]{variazione_auto_libere_enjoy_weekend}
\caption{Numero di auto libere Enjoy lungo l'arco della giornata}
\label{image:9}
\end{figure}

Per contestualizzare il tempo medio per raggiungere un'auto Enjoy è stato usato il dato sul numero delle macchine libere salvato per ogni query ai servizi. Il massimo numero di auto libere in circolazione è stato di 870. Nel grafico \ref{image:9} viene mostrata in media la variazione di questo conteggio lungo l'arco della giornata, ripartito per lunedì-venerdì e sabato-domenica. Si può notare come il picco di utilizzo del servizio, denotato da un calo delle auto libere a disposizione, inizia verso le 15:00 e finisce verso le 21:00 indipendentemente dal giorno. L'unica differenza tra i due gruppi si può notare nelle ore del mattino, che vede meno auto libere durante la settimana.

\todo{MM: devo inserire il grafico bruttino da vedere?}

Come è stato fatto con la velocità media per l'auto di proprietà, è stato calcolato il tempo medio per raggiungere un'auto libera Enjoy di settimana in settimana dalla fine del lockdown. Il risultato ottenuto è stato controintuitivo, mostrando come, col passare del tempo, il tempo medio per raggiungere un'auto libera diminuiva. Risulta controintuitivo infatti se si pensa al risultato ottenuto con la velocità media in auto, che ha visto una degradazione delle performance col ritorno alla normalità per quanto riguarda la mobilità. Analogamente si è pensato che il numero di utenti del servizio car sharing sarebbe aumentato col ritorno alla normalità e che quindi il tempo medio per raggiungere un auto sarebbe aumentato col diminuire delle auto libere in circolazione. Per investigare a fondo questo risultato è stato calcolato il numero di auto libere di settimana in settimana.

\begin{figure}[H]
\includegraphics[scale=1.0]{variazione_auto_libere_enjoy_weeks}
\caption{Numero di auto libere Enjoy lungo settimana dopo settimana}
\label{image:10}
\end{figure}

Il grafico \ref{image:10} mostra chiaramente come il numero delle auto libere a disposizione sia aumentato con la fine del lockdown. Si può notare infatti come siano state immesse circa 150 nuove auto nell'arco di un mese e altre 50 nel mese successivo. Questo dato di fatto sembra risolvere il risultato controintuitivo ottenuto nell'analisi precedente.


\subsection{Mezzi pubblici ATM, bici e a piedi}

\subsubsection{Velocità media di ora in ora}

\begin{figure}[H]
\includegraphics[scale=1.0]{vmedia_oraria_atm_bike_foot}
\caption{Variazione velocità media di ora in ora}
\label{image:11}
\end{figure}

Analizzando le stime di percorrenza riguardo i mezzi pubblici, la bicicletta e i percorsi a piedi, non si è verificata alcuna variazione significativa lungo l'arco della giornata, tradotto graficamente come nella figura \ref{image:11} in una retta parallela all'asse x per ognuno dei mezzi considerati. Il risultato ottenuto per la bicicletta e il percorso a piedi è stato ipotizzato, visto che tali mezzi non subiscono gli effetti del traffico nella stessa misura in cui li subiscono le auto e le moto. E' stata analizzata ulteriormente la performance dei mezzi pubblici, che vede una deviazione standard alta ma una performance pressochè costante a qualsiasi ora.

\begin{figure}[H]
\includegraphics[scale=1.0]{vmedia_oraria_atm_distance}
\caption{Variazione velocità media con ATM di ora in ora sulla distanza}
\label{image:12}
\end{figure}

Il grafico \ref{image:12} mostra lo stesso risultato a livello di variazione per qualsiasi lunghezza della tratta. Quel che varia è la velocità media, che vede una differenza di velocità tra le tratte brevi e lunghe.

\section{Confronto tra mezzi}

In questo paragrafo sono stati raccolti i risultati più interessanti emersi dal confronto tra tutti i mezzi di trasporto a disposizione per spostarsi nel Comune di Milano. Il confronto più atteso che ha fatto da guida verso questo studio è stato quello di verificare se un mezzo altamente costoso e inquinante come le auto a gasolio e benzina potessero essere surclassate da un mezzo più economico e pulito come i mezzi pubblici in una città dove il traffico influisce pesantemente sui tempi di percorrenza, come mostrato nel grafico \ref{image:3}.

\subsection{Vittoria dell'automobile}

Contro ogni aspettativa, nonostante la grande influenza del traffico sui tempi di percorrenza, i tragitti in auto sono risultati sempre i più veloci di ogni sua controparte, a qualsiasi ora del giorno e su ogni distanza, con una vittoria sopra il 99\% delle volte.

\subsection{Parziale sconfitta del car sharing}

A differenza del confronto con l'auto di proprietà, il confronto col servizio di car sharing si fa più interessante.

\subsubsection{Mezzi pubblici ATM vs. Enjoy}

\begin{table}[H]
\centering
\begin{tabular}{ | r r r | }
\hline
& \textbf{Abs. freq.} & \textbf{\% win} \\
\textbf{(2, 5] km} & 2072 & 4.2 \\
\textbf{(5, 7] km} & 1512 & 3.1 \\
\textbf{(7, 10] km} & 670 & 1.4 \\
\hline
\textbf{totale} & 4294 & 8.7 \\
\hline
\end{tabular}
\caption{Vittoria dei mezzi pubblici su car sharing per lunghezza tratta}
\label{table:5}
\end{table}

\begin{figure}[H]
\includegraphics[scale=0.8]{confronto_atm_enjoy}
\caption{Frequenza relativa vittorie ATM su Enjoy di ora in ora}
\label{image:13}
\end{figure}

I risultati del confronto tra i mezzi pubblici ATM e il servizio di car sharing Enjoy illustrati nella tabella \ref{table:5} mostrano una percentuale di vittorie di circa il 9\%, dove per vittoria si intende che il tempo impiegato a percorrere una tratta coi mezzi pubblici è stato minore o uguale al tempo impiegato utilizzando il car sharing. Nel grafico \ref{image:13} viene visualizzata la distribuzione di queste vittorie di ora in ora. Si può notare come vicino alle ore di picco del traffico individuate nell'analisi delle performance dell'auto, ovvero le ore 8:00-10:00 e 17:00-20:00, si concentri la percentuale maggiore delle vittorie.

Per esempio, su tutte le tratte richieste nelle ore 8:00, nell'11\% delle volte i mezzi pubblici hanno pareggiato o superato la performance del car sharing. Al contrario, al di fuori delle rush hours, la percentuale di vittorie è vicina allo 0. Dalla tabella \ref{table:5} si evince anche che dell'8.7\% di queste vittorie, circa metà di esse sono tratte brevi dai 2 ai 5 km.

\subsubsection{Bicicletta vs. Enjoy}

\begin{table}[H]
\centering
\begin{tabular}{ | r r r | }
\hline
& \textbf{Abs. freq.} & \textbf{\% win} \\
\textbf{(2, 5] km} & 15215 & 30.7 \\
\textbf{(5, 7] km} & 2423 & 4.9 \\
\textbf{(7, 10] km} & 278 & 0.6 \\
\hline
\textbf{totale} & 17917 & 36.2 \\
\hline
\end{tabular}
\caption{Vittoria della bicicletta su car sharing per lunghezza tratta}
\label{table:6}
\end{table}

\begin{figure}[H]
\includegraphics[scale=1.0]{confronto_bike_enjoy}
\caption{Frequenza relativa vittorie bicicletta su Enjoy di ora in ora}
\label{image:14}
\end{figure}

Risultato ancora più interessante è quello del confronto tra bicicletta di proprietà e car sharing. La tabella \ref{table:6} mostra una percentuale delle vittorie del 36\% sul totale, poco più di un terzo del totale delle tratte. La maggior parte di queste vittorie è concentrata nelle tratte brevi dai 2 ai 5 km.

Anche in questo confronto è emerso che nelle rush hours si accentua la percentuale di vittorie che arriva a toccare quasi il 45\% del totale nelle ore 8:00 e 18:00 come riportato dalla figura \ref{image:14}. Al contrario dei mezzi pubblici, questa percentuale resta alta anche al di fuori degli orari di punta, restando sempre intorno al 30\%.

\begin{figure}[H]
	\includegraphics[scale=1.0]{vmedia_oraria_all}
	\caption{Frequenza relativa vittorie bicicletta su Enjoy di ora in ora}
	\label{image:15}
\end{figure}

\begin{figure}[H]
	\includegraphics[scale=1.0]{vmedia_oraria_enjoy_weeks}
	\caption{Frequenza relativa vittorie bicicletta su Enjoy di ora in ora}
	\label{image:16}
\end{figure}

\begin{figure}[H]
	\includegraphics[scale=1.0]{vmedia_oraria_bici}
	\caption{Frequenza relativa vittorie bicicletta su Enjoy di ora in ora}
	\label{image:17}
\end{figure}

\begin{figure}[H]
	\includegraphics[scale=1.0]{vmedia_oraria_piedi}
	\caption{Frequenza relativa vittorie bicicletta su Enjoy di ora in ora}
	\label{image:17}
\end{figure}

\begin{figure}[H]
	\includegraphics[scale=1.0]{vmedia_oraria_atm}
	\caption{Frequenza relativa vittorie bicicletta su Enjoy di ora in ora}
	\label{image:17}
\end{figure}

















