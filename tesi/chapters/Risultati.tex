\section{Raccolta dati}

I dati riguardanti la simulazione sono stati raccolti a partire dall'1 marzo 2020 fino al 27 giugno 2020 per un arco di tempo di circa 4 mesi, al ritmo di 1,000 richieste al giorno. L'Italia è entrata in uno stato di lockdown nazionale il 12 marzo e ha allentato le restrizioni a partire dal 4 maggio. I dati raccolti a partire da quest'ultima data sono considerati validi per le analisi. Sono stati salvati in formato .csv, dove ogni riga contiene le seguenti informazioni:

\begin{center}
	\begin{tabular}{ c c c c c c c c c  }
		Data & Orario & Partenza & Arrivo & Auto & ATM & Enjoy & Bici & Piedi
	\end{tabular}
\end{center}

In partenza e arrivo sono salvate le coordinate espresse in gradi del tragitto generato, nei restanti campi le stime di percorrenza di tale tragitto espresse in minuti, insieme alla data e all'orario in cui è stata effettuata la richiesta di tale tratta.

Oltre ai dati principali sono stati salvati informazioni secondarie come il numero di auto libere Enjoy al momento della richiesta, la lunghezza aerea della tratta e la lunghezza calcolata a piedi della stessa, entrambe espresse in km.

\section{Analisi dati raccolti}

\subsection{Filtro dati errati}

\begin{table}
	\centering
	\begin{tabular}{ | l r r | }
		\hline
		& \textbf{n\textdegree tratte} & \textbf{\%} \\ 
		\textbf{Coerenti} & 49560 & 73 \\  
		\textbf{Errati} & 18290 & 27 \\
		\hline
		\textbf{Totale} & 67850 & 100 \\
		\hline
	\end{tabular}
	\caption{I dati errati contengono uno o più zeri nelle stime}
	\label{table:1}
\end{table}

I provider per le stime di percorrenza dei vari mezzi hanno talvolta restituto il valore zero. Questo fenomeno è probabilmente dovuto alla generazione a random delle coordinate di partenza e arrivo, che ha permesso richieste in qualsiasi punto della mappa, compresi punti non situati direttamente sulla strada come in giardini pubblici, condimini e altre zone private, o semplicemente per disservizi. Per un confronto alla pari sono state considerate solo le righe prive di zeri nel file CSV. Ne è risultato che su circa 68000 richieste, circa 50000 sono prive di zeri, circa il 73\% del totale.

\subsection{Distribuzione tratte generate}

\begin{center}
\includegraphics[scale=0.8]{distribuzione_tratte}
\end{center}

\begin{table}
	\centering
	\begin{tabular}{ | l r | }
		\hline
		\textbf{Abs. freq.} & 49560 \\
		\textbf{media} & 3.80 \\
		\textbf{mediana} & 3.57 \\
		\textbf{std} & 1.27 \\
		\textbf{min} & 2.00 \\
		\textbf{max} & 9.52 \\
		\hline
	\end{tabular}
	\caption{I dati errati contengono uno o più zeri nelle stime}
	\label{table:1}
\end{table}

Le tratte generate a random sono distribuite intorno a una media di 3.8km con una deviazione standard di circa 1.3km.

















