\section{Uso dell'automobile in città}

La città di Milano è in costante espansione. Sempre più nuove aree vengono progettate, realizzate e aggretate all'interno del suo complesso urbanistico. Battezzata come la capitale del lavoro, vanta la presenza di numerosi settori lavorativi, dall'industriale al digitale, e svariati servizi. Attualmente, risulta al secondo posto tra le città italiane per numero di abitanti e per densità di abitanti per chilometro quadrato.

Sono queste le principali forze che hanno accelerato lo sviluppo della rete stradale e della rete ATM, l'Azienda Trasporti Milanesi. Molti infatti sono i cambiamenti apportati alla città per fronteggiare la domanda crescente di spostamenti e per diminuire il traffico in determinate aree, come per esempio la realizzazione della Linea M5, l'introduzione dell'Area C e delle numerose Zone a Traffico Limitato (ZTL), l'aumento delle strade a corsia preferenziale e la creazione di piste ciclabili. Ultimamente ha anche preso piede l'uso della mobilità condivisa, che vede diverse aziende nel settore distribuire i propri mezzi, quali automobili, monopattini elettrici e biciclette a pedalata assistita, all'interno del Comune di Milano, noleggiando a tempo tali mezzi. Uno dei problemi principali che riguarda l'intero territorio italiano è il numero di automobili in circolazione. Questo mezzo di trasporto è ormai più diffuso della bicicletta. Da diversi anni l'Italia è il secondo Paese in Europa col maggior numero di auto per abitanti, con circa 640 auto ogni 1000 abitanti\cite{eurostatcars}, e, nello specifico, 511 ogni 1000 nella sola città di Milano. Inoltre, nell'ultimo studio annuale di TomTom del 2019\cite{tomtomindexmilan}, Milano rientra tra le prime 110 città su scala globale per livello di congestione stradale, con un traffico tale da far aumentare di quasi il 50\% il tempo impiegato a percorrere una tratta durante le ore di picco.

A fronte di questi dati, viene spontaneo chiedersi quanto sia influenzato il tempo di percorrenza in automobile su strade cittadine e, soprattutto, se questa influenza sia tale da favorire altri mezzi di trasporto. Sebbene sia un dato di fatto che, su strada, un'automobile sia più veloce di un tram o di una bicicletta, dall'altra bisogna ricordare dei vantaggi di cui godono gli altri mezzi. Basti pensare alle numerose strade con corsia preferenziale o carreggiata a parte per bus e tram, alle lunghe piste ciclabili che ultimamente sono state create per favorire la mobilità alternativa, fino a mezzi che non soffrono minimamente dei problemi del traffico, come le metropolitane ATM e i passanti ferroviari Trenord. L'obbiettivo dello studio è quello di analizzare a livello di tempistiche l'impiego dell'automobile di proprietà all'interno di Milano e confrontarlo con mezzi alternativi per gli spostamenti al suo interno, al fine di capire se l'influenza delle congestioni che caratterizzano le strade di città medio-grandi.

%Di queste, lo 0.4\% sono elettriche\cite{anfiastudiestatistiche}. 

\section{Problemi}

Questo mezzo di trasporto, sfortunatamente, ha diversi problemi. Primo tra tutti è sicuramente l'\textbf{inquinamento}, che viene affrontato in tutte le fasi della vita di un'automobile, dagli standard dell'Unione Europea sulle emissioni a cui si devono attenere i produttori di auto per poter vendere sul territorio, alle aziende di idrocarburi, che si impegnano a introdurre una percentuale di biocarburante nelle miscele, e ancora dalle politiche territoriali, come l'introduzione di aree a divieto di transito per veicoli inquinanti di una certa categoria, fino allo smaltimento del veicolo stesso, sempre regolato a livello legislativo. Nonostante queste numerose restrizioni i risultati stentano ad arrivare. I dati del 2018 dell'Automobile Club d'Italia (ACI) riportano come il 63\% delle auto in circolazione sia di categoria Euro IV o minore\cite{anfiastudiestatistiche}, dove Euro IV è uno standard sulle emissioni introdotto nel 2006 e superato nel 2008 dall'Euro V, ben 12 anni fa\cite{euroivstandard}. Nel periodo 2008-2017 compresi, l'Italia ha superato in modo continuato i limiti giornalieri e annuali di livello di polveri sottili PM10, e per questo è stata multata dalla Corte di Giustizia dell'Unione Europea a pagare una cospicua multa\cite{eunewssanzioneitalia}. Se visto in prospettiva, l'inquinamento prodotto dalle auto non è una percentuale rilevante dell'inquinamento dell'aria, difatti è emerso in numerosissimi studi effettuati in diverse città del mondo, tra cui Milano e Brescia\cite{collivignarelli2020}\cite{camaletti2020}, durante periodo locale di lockdown per l'emergenza da COVID19, che, anche a fronte di un traffico pressochè nullo, nell'aria si sono registrati gli stessi livelli di sostanze chimiche dannose registrati nei periodi precedenti in condizioni di traffico abituale, con lievi abbassamenti riguardanti le sostanze riconducibili direttamente alla combustione dei motori termici. Questi studi infatti suggeriscono che la maggior parte dell'inquinamento dell'aria è legata a fattori ambientali. Una conclusione simile è stata ottenuta analizzando i dati dell'inquinamento dell'aria nelle aree dove è stata introdotta una restrizione sulla circolazione di alcuni veicoli, come l'area C nel comune di Milano (ex Ecopass), che concretamente ha dato scarsi risultati\cite{trentini2014}. Nonstante questo però, non bisogna dimenticare che i gas di scarico derivanti dai motori diesel risultano sicuri cancerogeni per l'uomo, al pari del fumo di tabacco, con la sottile differenza che quest'ultimo è qualcosa che si può evitare\cite{iarctable}. L'inquinamento dunque rimane un problema, dal momento che è maggiormente presente nelle città e negli agglomerati urbani, dove è presente la maggior parte della popolazione e dove le automobili, a livello di consumo di combustibile, risultano meno efficienti per via delle basse velocità.

Secondo problema dell'auto, se comparato con gli altri mezzi di trasporto per uso cittadino, è il \textbf{costo}. Una stima svolta da SosTariffe nel 2019 mostra come mantenere un'auto in Italia costi circa €1,620 all'anno, compresi di assicurazione, benzina e bollo auto, esclusi il prezzo del mezzo stesso, che si aggira intorno ai €11,000 per un'utilitaria a chilometro zero, il cambio delle gomme ed eventuali imprevisti\cite{sostariffe}. Seppur questo problema risulti molto relativo, dato che è il reddito e il patrimonio di una persona a incidere sulla percezione di questa spesa, il solo costo di mantenimento di un'auto risulta molto più dispendioso rispetto ai mezzi alternativi, come un abbonamento annuale per i mezzi pubblici, per treni, o come l'acquisto e il mantenimento di una bicicletta o di mezzi alternativi.

Ultimo ma non ultimo dei problemi è l'\textbf{efficienza} dell'auto in termini di tempo all'interno delle grandi città, come già citato precedentemente. Tra le proprietà che caratterizzano lo spostamento in auto vi è sicuramente l'ingombro stradale rapportato al numero di persone in viaggio, ovvero l'utilizzo dell'automobile per lo spostamento del solo conducente, nonostante la capacità maggiore dell'auto, che congestiona le strade per un flusso di persone relativamente minore di persone rispetto a un bus. A documentarlo è l'Associazione Nazionale Comuni Italiani (ANCI), che ha stimato che, degli 1.8 milioni di veicoli che si spostano quotidianamente nelle città italiane per tragitti casa-lavoro o casa-studio, 1.2 milioni viaggiano con solo il conducente a bordo\cite{anciperrepubblica}, più del 66\%.

\section{Vantaggi}

A giustificare la grande diffusione di questo mezzo di trasporto tra la popolazione sono sicuramente i grandi vantaggi che l'uso o il possesso di un auto comportano. Prima di tutto, possedere un auto può essere considerato sinonimo di libertà di muoversi: non si hanno limiti di orario di partenza, perchè l'auto è pronta per essere usata a qualsiasi ora. La decisione dell'orario di inizio viaggio è esclusivamente del conducente, al contrario dei mezzi pubblici dove l'orario di partenza è offerto in numero limitato e prestabilito, con scarse percorrenze negli orari notturni; non si hanno limiti di distanza, perchè in automobile si può percorrere qualsiasi distanza senza particolari sforzi fisici, al contrario della bicicletta; non si hanno limiti di destinazione, perchè essenzialmente con l'automobile si può raggiungere qualsiasi posto raggiungibile con i mezzi alternativi e non; si possono trasportare merci molto pesanti, difatti uno tra gli impieghi principali dell'auto, dopo quello del raggiungimento del luogo di lavoro e/o di studio, è quello di fare la spesa al supermercato e quello legato al trasporto di merce ingombrante acquistata, difficile da trasportare in assenza di un veicolo e spesso delegata a chi fornisce servizi di spedizioni. Altri vantaggi legati all'auto sono legati alla sua comodità e praticità. Per esempio, nel Comune di Milano, la maggioranza dei parcheggi sono riservati ai residenti e sono divisi a zone, garantendo la possibilità di trovare parcheggio nei pressi della propria abitazione, sebbene tale provvedimento sia stato pensato originariamente a ostacolare gli spostamenti in auto all'interno della città. Inoltre, in caso di pioggia, temporale e persino neve, viaggiare in auto risulta l'opzione più comoda e sicura, talvolta l'unica funzionante.

\section{Confronto con mezzi alternativi}

È sul terzo problema che si concentra questo studio, che ha come obbiettivo capire quanto sia vantaggioso o svantaggioso in termini di tempi di percorrenza usare un'automobile all'interno della città di Milano, a fronte di problemi come traffico e congestioni stradali, da parte di una persona con pure esigenze di spostamento, senza vincoli legati a trasporto merci, intemperie o raggiungimento di luoghi al di fuori della città. Per capire quale sia il mezzo più vantaggioso, la soluzione ideale sarebbe quella di organizzare diverse gare in diversi punti della città, in cui delle persone, ognuno con un mezzo di trasporto diverso, partono da un comune punto di partenza e raggiungono un comune punto di destinazione, cronometrando il tempo impiegato da ciascuno. Purtroppo, non avendo a disposizione le risorse e il tempo (e, date le restrizioni per la pandemia di COVID19, nemmeno il permesso) per effettuare un confronto del genere, si è optato per una soluzione informatica.
